\section{Introduction}

This is an attempt to formalize the \textsc{ieee}-754 standard
(specifically, the 2019 revision) in Mizar. The goal is to faithfully
implement the standard, while allowing suitable generalities so we could
work with any base $\beta\in\RR$ provided $\beta>1$.

For ease of reference, the sections in this literate proof correspond to
those in the Standard, with the following exceptions:
\begin{itemize}
\item We started with Section 0 for this informal preamble, whereas the
  Standard starts with Section 1;
\item Section 1 of the Standard consists of an overview, whereas Section
  1 of this formalization consists of preliminary definitions; and
\item Section 2 of the Standard is a glossary of terms and acronyms.
\end{itemize}
That's it.

\subsection{Apology: Theorem numbering}
We will be following a ``five counter theorem numbering'' system, which
will follow Mizar's numbering conventions. This is to make reference
between the informal presentation and the Mizar formalization easier. It
is quite bewildering, and I can only apologize for it. The four counters
are independent of each other, and are grouped roughly as follows:
\begin{enumerate}
\item Theorem and Corollary share the same counter;
\item Lemmas have their own counter;
\item Schemes have their own counter;
\item Definitions and redefinitions have their own counter;
\item All other environments are either informal [propositions, idioms,
  examples, etc.] or unnumbered in Mizar [registrations, reservations,
  vocabulary declarations, abbreviations, etc.], so I lump them all
  together with the same counter.
\end{enumerate}

\subsection*{Dedication}
I'd like to dedicate this to Roland Freund, Angela Cheer, and Elbridge
Gerry Puckett, all of whom introduced me to the wonderful world of
numerical analysis. ``The old fox never forgets the hillock where he was
born; the white turtle repaid the kindness he had received from Mao
Pao. If even animals will do such things, then how much more so should
this be true of human beings?''
%% ``The old fox never forgets the hillock where he was
%% born; the white turtle repaid the kindness he had received from Mao
%% Pao. If even lowly creatures know enough to do this, then how much more
%% should human beings! What can we say, then, of persons who are devoting
%% themselves to study? Surely they should not forget the debts of
%% gratitude they owe to their parents, their teachers, and their
%% country.''
